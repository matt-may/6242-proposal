% This is "sig-alternate.tex" V2.1 April 2013
% This file should be compiled with V2.5 of "sig-alternate.cls" May 2012
%
% This example file demonstrates the use of the 'sig-alternate.cls'
% V2.5 LaTeX2e document class file. It is for those submitting
% articles to ACM Conference Proceedings WHO DO NOT WISH TO
% STRICTLY ADHERE TO THE SIGS (PUBS-BOARD-ENDORSED) STYLE.
% The 'sig-alternate.cls' file will produce a similar-looking,
% albeit, 'tighter' paper resulting in, invariably, fewer pages.
%
% ----------------------------------------------------------------------------------------------------------------
% This .tex file (and associated .cls V2.5) produces:
%       1) The Permission Statement
%       2) The Conference (location) Info information
%       3) The Copyright Line with ACM data
%       4) NO page numbers
%
% as against the acm_proc_article-sp.cls file which
% DOES NOT produce 1) thru' 3) above.
%
% Using 'sig-alternate.cls' you have control, however, from within
% the source .tex file, over both the CopyrightYear
% (defaulted to 200X) and the ACM Copyright Data
% (defaulted to X-XXXXX-XX-X/XX/XX).
% e.g.
% \CopyrightYear{2007} will cause 2007 to appear in the copyright line.
% \crdata{0-12345-67-8/90/12} will cause 0-12345-67-8/90/12 to appear in the copyright line.
%
% ---------------------------------------------------------------------------------------------------------------
% This .tex source is an example which *does* use
% the .bib file (from which the .bbl file % is produced).
% REMEMBER HOWEVER: After having produced the .bbl file,
% and prior to final submission, you *NEED* to 'insert'
% your .bbl file into your source .tex file so as to provide
% ONE 'self-contained' source file.
%
% ================= IF YOU HAVE QUESTIONS =======================
% Questions regarding the SIGS styles, SIGS policies and
% procedures, Conferences etc. should be sent to
% Adrienne Griscti (griscti@acm.org)
%
% Technical questions _only_ to
% Gerald Murray (murray@hq.acm.org)
% ===============================================================
%
% For tracking purposes - this is V2.0 - May 2012

\documentclass{sig-alternate-05-2015}


\begin{document}

% Copyright
\setcopyright{acmcopyright}
%\setcopyright{acmlicensed}
%\setcopyright{rightsretained}
%\setcopyright{usgov}
%\setcopyright{usgovmixed}
%\setcopyright{cagov}
%\setcopyright{cagovmixed}


% DOI
\doi{10.475/123_4}

% ISBN
\isbn{123-4567-24-567/08/06}

%Conference
% % % % \conferenceinfo{PLDI '13}{June 16--19, 2013, Seattle, WA, USA}

% % % % \acmPrice{\$15.00}

%
% --- Author Metadata here ---
% % % % \conferenceinfo{WOODSTOCK}{'97 El Paso, Texas USA}
%\CopyrightYear{2007} % Allows default copyright year (20XX) to be over-ridden - IF NEED BE.
%\crdata{0-12345-67-8/90/01}  % Allows default copyright data (0-89791-88-6/97/05) to be over-ridden - IF NEED BE.
% --- End of Author Metadata ---

\title{Risk Analysis of Skin Cancer using Deep Learning}
\subtitle{[An Online system to augment clinician's decision making ]
\titlenote{Work done as part cs-6242 class at Georgia Tech}}
%
% You need the command \numberofauthors to handle the 'placement
% and alignment' of the authors beneath the title.
%
% For aesthetic reasons, we recommend 'three authors at a time'
% i.e. three 'name/affiliation blocks' be placed beneath the title.
%
% NOTE: You are NOT restricted in how many 'rows' of
% "name/affiliations" may appear. We just ask that you restrict
% the number of 'columns' to three.
%
% Because of the available 'opening page real-estate'
% we ask you to refrain from putting more than six authors
% (two rows with three columns) beneath the article title.
% More than six makes the first-page appear very cluttered indeed.
%
% Use the \alignauthor commands to handle the names
% and affiliations for an 'aesthetic maximum' of six authors.
% Add names, affiliations, addresses for
% the seventh etc. author(s) as the argument for the
% \additionalauthors command.
% These 'additional authors' will be output/set for you
% without further effort on your part as the last section in
% the body of your article BEFORE References or any Appendices.

\numberofauthors{4} %  in this sample file, there are a *total*
% of EIGHT authors. SIX appear on the 'first-page' (for formatting
% reasons) and the remaining two appear in the \additionalauthors section.
%
\author{
% You can go ahead and credit any number of authors here,
% e.g. one 'row of three' or two rows (consisting of one row of three
% and a second row of one, two or three).
%
% The command \alignauthor (no curly braces needed) should
% precede each author name, affiliation/snail-mail address and
% e-mail address. Additionally, tag each line of
% affiliation/address with \affaddr, and tag the
% e-mail address with \email.
%
% 1st. author
\alignauthor
Ben Trovato\\
       \affaddr{Institute for Clarity in Documentation}\\
       \affaddr{1932 Wallamaloo Lane}\\
       \affaddr{Wallamaloo, New Zealand}\\
       \email{trovato@corporation.com}
% 2nd. author
\alignauthor
Matthew May\titlenote{Matt's inspiration is his father who was diagnosed with melanoma in 2015}\\
       \affaddr{Georgia Institute of Technology}\\
       \email{maym@gatech.edu}
% 3rd. author
\alignauthor Lars Th{\o}rv{\"a}ld\\
       \affaddr{The Th{\o}rv{\"a}ld Group}\\
       \affaddr{1 Th{\o}rv{\"a}ld Circle}\\
       \affaddr{Hekla, Iceland}\\
       \email{larst@affiliation.org}
\and  % use '\and' if you need 'another row' of author names
% 4th. author
\alignauthor Lawrence P. Leipuner\\
       \affaddr{Brookhaven Laboratories}\\
       \affaddr{Brookhaven National Lab}\\
       \affaddr{P.O. Box 5000}\\
       \email{lleipuner@researchlabs.org}
}


\maketitle
\begin{abstract}
Skin cancer is the most common form of cancer, accounting for 40\% of cases globally.
More than 8,500 people in the US are diagnosed with skin cancer everyday. Melanoma rates in the United States have doubled from 1982 to 2011.
The survival rate of people with it depends on when they start treatment. The cure rate is very high
with early prognosis. The curability is as high as 92\% if the cancer is detected early. \cite{prognosticfactors} The prognosis is less favorable 
if it has spread to other parts of the body.
Deep convolutional neural networks have led to new breakthroughs in image classification.
Advancement in neural network frameworks and GPU based computing has allowed us to train deeper and deeper networks.
We aim to exploit these advancements and apply it to the problem of early skin cancer detection. We achieve an overall accuracy of 90\% for the
binary classification of malignant and benign skin lesions and an accuracy of 40\% for multi-class cancer type classification.

\end{abstract}




% We no longer use \terms command
%\terms{Theory}

\keywords{Deep Learning; Neural Networks; Skin-cancer classification}

\section{Introduction}
After the onset of a lesion patients wait several months for an appointment with the doctor since these are usually
in the form of mild moles. Dermatologists usually have no way to follow the lesions of their patients over time
outside their clinic. \\
In this paper we describe, \url{https://dermfollow.me/}, a system we built to alleviate this problem. Patients can
use smartphones to upload pictures of their lesions on the website; say, every week. The system uses the knowledge learnt
from several thousand images using deep neural networks to calculate the risk score for the uploaded image and present the
most likely cases and high risk patients before the clinician. \\
We also present the similar images to explain why the neural network considers an image as a high risk case. The clinician can use this added
information and schedule a speedy appointment with these patients on a priority basis.



\section{Related Work}
It is only recently that neural networks have been used for the problem of skin cancer classification. Much of the earlier work focussed
on extracting features based on image processing techniques to do the classification. In \cite{DBLP:journals/eswa/LeeC15}, Lee and Chen use low
level feature comparison for detecting different cancer types. In \cite{Kreutz2001}, the authors use a combination of neural networks and 
sophisticated image processing and feature extraction techniques to achieve a fast and reliable diagnosis. They were able to obtain a 
90\% correct classification of malignant and benign skin lesions from the DANAOS data collection. In \cite{Sheha_automaticdetection} the authors use 
a multilayer perceptron to classify melanoma.

There have also been techniques like \cite{Gniadecka2004443}, \cite{Zhao:15} which combine spectroscopy data with neural networks. These techniques
are only plausible in a lab setting with appropriate instruments. The novelty of our technique is that it relies solely on image data.
These images can be captured by an individual using smartphones and does not require any specialized pathology skills. The user could get a reminder
to upload the image to the website on a periodic basis and the dermatologist could monitor high risk patients easily using our system.\\
The most recent work which is related to our current work was done by Andre, Kuprel and Sebastian Thrun at Stanford. \cite{esteva-skincancer-manuscript}.
The approach used by them is an ensemble of 5 models (VGG16-1, VGG16-2, VGG16-3, VGG19-1, and VGG19-2) that classify among the 23 classes. We try to
achieve a better accuracy over their model and also build a system for dermatologists to efficiently manage their patients.

\section{Approach}
ask stefano about this

\subsection{Datasets}
You may want to display math equations in three distinct styles:
inline, numbered or non-numbered display.  Each of
the three are discussed in the next sections.

\subsubsection{Tensorflow}
A formula that appears in the running text is called an
inline or in-text formula.  It is produced by the
\textbf{math} environment, which can be
We use tensorflow \cite{tensorflow2015-whitepaper}


\section{Design}
Mention about salient features and the design choices you made

\subsection{Architecture}
the overall architecture of the website


\section{Experiment and Results}
asdsd




\section{Conclusions}
This paragraph will end the body of this sample document.
Remember that you might still have Acknowledgments or
Appendices; brief samples of these
follow.  There is still the Bibliography to deal with; and
we will make a disclaimer about that here: with the exception
of the reference to the \LaTeX\ book, the citations in
this paper are to articles which have nothing to
do with the present subject and are used as
examples only.
%\end{document}  % This is where a 'short' article might terminate


%ACKNOWLEDGMENTS are optional
\section{Acknowledgments}
This section is optional; it is a location for you
to acknowledge grants, funding, editing assistance and
what have you.  In the present case, for example, the
authors would like to thank Gerald Murray of ACM for
his help in codifying this \textit{Author's Guide}
and the \textbf{.cls} and \textbf{.tex} files that it describes.

%
% The following two commands are all you need in the
% initial runs of your .tex file to
% produce the bibliography for the citations in your paper.
\bibliographystyle{abbrv}
\bibliography{sigproc}  % sigproc.bib is the name of the Bibliography in this case
% You must have a proper ".bib" file
%  and remember to run:
% latex bibtex latex latex
% to resolve all references
%
% ACM needs 'a single self-contained file'!
%
%APPENDICES are optional
%\balancecolumns




%\balancecolumns % GM June 2007
% That's all folks!
\end{document}
