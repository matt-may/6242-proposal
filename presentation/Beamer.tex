\documentclass[14pt]{beamer}
% \useoutertheme{miniframes}

\usepackage{ngerman}
\usepackage[utf8]{inputenc}
\usepackage[T1]{fontenc}
\usepackage{textcomp}
% \usepackage{amsmath,amsfonts,amssymb}
\usepackage{url}
\usepackage{commath}

% disable Navigation at the bottom
\beamertemplatenavigationsymbolsempty

% page numbers
% \setbeamertemplate{footline}[frame number]
% \setbeamertemplate{footline}
\setbeamertemplate{headline}

% style for source code
\usepackage{listings}
\newcommand{\Hilight}{\makebox[0pt][l]{\color{light-gray}\rule[-4pt]{1.0\linewidth}{12pt}}}
\usepackage[framed,numbered]{mcode}
\usepackage{color}
\definecolor{light-gray}{gray}{0.80}
  
% use {\mono lorem} to set something in monospace
\newcommand{\mono}[1]{\ttfamily\fontsize{14}{14}\selectfont #1}

% wanna use Metapost?
\makeatletter
\newcommand\@ptsize{12}
\makeatother

\usepackage{mflogo}
\usepackage{emp}

\DeclareGraphicsRule{*}{mps}{*}{}


\title{Risk Analysis of Skin Cancer using Deep Learning}   
\author{Matthew May, Stefano Fenu, Thanh Dang, Apurv Verma} 
\date{\today} 


%===================================
\begin{document}

\frame{\titlepage} 

% \frame{\frametitle{Table of contents}\tableofcontents} 


%===================================

\frame{\frametitle{}
	\begin{figure}[ht]
		\centering{
			\Large{Motivation}
		}
	\end{figure}
}

\section{Introduction} 
\subsection{Motivation}

\frame{\frametitle{Motivation} 
	\begin{itemize}
		\item Skin cancer is the most common form of cancer, accounting for 40\% of cases globally. More than 8,500 people
		in the US are diagnosed with skin cancer everyday.\\ \ \\
		
		\item The cure rate is very high with early prognosis. The curability is as high as 92\% if the cancer is detected early.\\ \ \\
		
		\item Unfortunate fact of the healthcare system that it spends more in research on expensive treatments for late-stage diseased individuals
than developing scalable and cost-effective early screening methods.\\ \ \\
	\end{itemize}
}

\frame{\frametitle{}
	\begin{figure}[ht]
		\centering{
			\Large{Problem Statement}
		}
	\end{figure}
}

\frame{\frametitle{Problem Statement} 
	\begin{itemize}
		\item To build a platform where dermatologists can easily scan through the cancer
prone patients. The system uses deep learning techniques in the background
to learn from thousands of images and associates a risk score with each uploaded
image. The value of the score can be used to differentiate between malign and
benign cases. We also aim to train a model for multi-class classification among
the 23 classes of skin cancer.
	\end{itemize}
}


\frame{\frametitle{}
	\begin{figure}[ht]
		\centering{
			\Large{Design and Approach}
		}
	\end{figure}
}

\frame{\frametitle{Design and Approach} 
	\begin{itemize}
		\item Experiment with state of the art image classification models like Google Net, ResNet, Net2Net for classification.
		\item Build an intuitive web application where patients can upload images of their lesions over time
		\item The application provides an interface to dermatologists to monitor patients outside clinical settings.
		\item The application also provides insights from knowledge learnt through thousands of images to suggest high risk patients
		and similar lesions.
		\item The dermatologist can call in the patient if it's a high risk case.
		
	\end{itemize}
}







%===================================
% \section{Content} 
% \subsection{First Content}
% \frame{\frametitle{Some stuff}
% 	\begin{itemize}
% 		\item Code
% 		\item Algorithmen entwerfen
% 		\item Parameter \& Codeanpassung
% 		\item Archtitektur, Produktdesign
% 	\end{itemize} 
% }
% 
% 
% \frame{\frametitle{API}
% 	\begin{itemize}
% 		\item 2D / 3D Primitives
% 		\item PFont, PImage, \dots{}
% 		\item Input, Output\\ \ \\ 
% 		
% 		\item + enorm viele Librarys
% 		\item + viele Beispiele
% 	\end{itemize} 
% }
% 
% 
% 
% \begin{frame}[fragile]
% 	\frametitle{Some Code}
% 
% 	\begin{lstlisting}[language=Java, frame=single,
% 				% width of the box: linewidth=6.5cm, 
% 				basicstyle=\ttfamily\fontsize{10}{12}\selectfont,
% 				escapechar=\%]
% 		%\Hilight%void setup() 
% 		{
% 		  size(250, 250); 
% 		  smooth();
% 		}
% 				
% 		void draw() 
% 		{
% 		  background(0);
% 		  ellipse(width/2, height/2, 
% 		          30, 30);
% 		}		
% 	\end{lstlisting}	
% \end{frame}
% 
% 
% \begin{frame}[fragile]
% 	\frametitle{Some Code}
% 
% 	\begin{lstlisting}[language=Java, frame=single,
% 				basicstyle=\ttfamily\fontsize{10}{12}\selectfont,
% 				escapechar=\%]
% 		void setup() 
% 		{
% 		%\Hilight%  size(250, 250); 
% 		  smooth();
% 		}
% 				
% 		void draw() 
% 		{
% 		  background(0);
% 		  ellipse(width/2, height/2, 
% 		          30, 30);
% 		}		
% 	\end{lstlisting}	
% \end{frame}
% 
% 
% 
% \begin{frame}[fragile]
% 	\frametitle{\MP}
% 	\begin{figure}[ht]
% 		\centering{
% 			\includegraphics{content/house.mps}
% 		}
% 	\end{figure}
% \end{frame}
% 
% 
% \frame{\frametitle{}
% 	\begin{figure}[ht]
% 		\centering{
% 			\Large{\texttt{\$ traceroute}}
% 		}
% 	\end{figure}
% }
% 
% \frame{\frametitle{Was brauchen wir?}
% 	\begin{itemize}
% 		\item Stream: {\mono traceroute} nach Processing		
% 		\item Separater Thread im Hauptprogramm
% 	\end{itemize} 
% }
% 
% 
% \frame{\frametitle{}
% 	\begin{figure}[ht]
% 		\centering{
% 			\Large{Website $\rightarrow$ Dynamische Datenstruktur}
% 		}
% 	\end{figure}
% }
% 
% 
% \frame{\frametitle{Wikileaks}
% 	\begin{itemize}
% 		\item 28. November 2010: Diplomatic cable release
% 		\item 2. Dezember 2010: EveryDNS\\ \ \\		
% 		
% 		\item ``\textit{Wikileaks is currently under heavy attack.}''
% 	\end{itemize}
% }
% 
% 
% %===================================
% \section{The End}
% \frame{\frametitle{Weiterführendes} 
% 
% 	\begin{block}{Online}
% 	\begin{itemize}
% 		\item \url{http://processing.org/reference}
% 		\item \url{http://nodejs.org}
% 	\end{itemize}
% 	\end{block}
% 
% 	\begin{block}{Bücher}
% 	\begin{itemize}
% 		\item Generative Gestaltung: Entwerfen. Programmieren. Visualisieren.
% 		\item The OpenGL Programming Guide
% 	\end{itemize}
% 	\end{block}
% }
% 
% \frame{\frametitle{Software used} 
% 	\begin{block}{Umgebung}
% 		Mac OS X, vim
% 	\end{block}
% 
% 	\begin{block}{Satz}
% 		\LaTeX{} beamer
% 	\end{block}
% 
% 	\begin{block}{Grafiken}
% 		METAPOST
% 	\end{block}
% 	
% 	\vfill
% 	Slides are available on \url{http://github.com/cmichi}
% }


\end{document}

