\documentclass[a4paper,12pt]{article}
\usepackage[utf8]{inputenc}
\usepackage[colorlinks]{hyperref}
\usepackage{epstopdf}
\hypersetup{
  colorlinks = true,
  linkcolor  = magenta,
  citecolor  = magenta
}
\usepackage[all]{hypcap}

\title{DermFollow: A System For Better Patient Follow-Up In Dermatology}
\author{Thanh Dang, Stefano Fenu, Matthew May, Apurv Verma}

\begin{document}

\maketitle

\section{What Are You Trying To Do?}
Our goal is to build a Web-based system that allows dermatologists to more
effectively follow up with patients that have, or are at risk for, skin cancer.
Our system will do this by providing classification of suspected skin cancer
lesions, patient risk scoring, and patient image analysis over time. Ultimately,
we hope to improve patient outcomes, raise patient satisfaction, and improve
the efficiency of medical practice.

\section{How Is It Done Today?}
The current state-of-the-art for automated skin cancer analysis generally
involves feature detection \cite{DBLP:journals/eswa/LeeC15}. In dermatology,
there is a widely used ABCDE \cite{thomas1998semiological} algorithm that is
used by doctors to assess potential skin cancer lesions. Existing approaches
often attempt to imitate this algorithm via detection of the same features
\cite{zagrouba2004prelimary}. However, the limitation of many of these models
has been low-quality results on the more diverse images actually encountered
in practice, limiting adoption.

Relatively recently, neural networks have been applied to skin cancer
classification. In \cite{Kreutz2001}, the authors use a combination of neural
networks and feature extraction to classify skin lesions. Sheha et al.
\cite{Sheha_automaticdetection} use a multilayer perceptron to classify
melanoma, attaining 92\% accuracy. Esteva et al.
\cite{esteva-skincancer-manuscript} use an ensemble of convolutional neural
networks (CNNs) to attain 90\% binary classification (malignant/benign).
While these results are promising, many of these models were trained on images
that lack histological (microscopic) verification, the gold standard for
determining malignancy.

\section{What's New In Your Approach? Why Will It Succeed?}
Low patient engagement, limited integration into clinical workflows, poor
algorithm performance on actual patient images, and expensive hardware
requirements have plagued existing approaches.

To increase patient engagement, our system will provide patients rewards for
uploading images of their lesions over time, providing an incentive for
compliance.

Our system will integrate easily with existing electronic clinical workflows.
It will perform analysis on uploaded patient images and then present the results
to the physician in an uncluttered way, allowing immediate patient follow-up.

To improve algorithm performance, we will use state-of-the-art CNNs, which have
been shown to be very effective at image classification
\cite{krizhevsky2012imagenet}. By allowing patients to upload images from
their smartphone, no additional hardware will be required. Lastly, we use
datasets for training and testing with 100\% histological verification.

\section{Who Cares?}
It has been shown that health care IT can lead to significant increases in
patient satisfaction \cite{roham2012predicting}. Patient satisfaction is one of
the primary metrics now considered an indicator of health care quality by
hospitals \cite{fenton2012cost}, \cite{bjertnaes2012overall}.

By allowing patients to receive a higher standard of care through technology, we
believe patient satisfaction will grow. The literature has shown that
telemedicine technology is effective \cite{hilty2013effectiveness}. By providing
physicians with meaningful data and analysis, we believe patient outcomes and
satisfaction will increase.

\section{What Difference Will It Make?}
To determine whether our approach would have utility in practice, we spoke with
several dermatologists. One common thread was that patient follow-up is limited.
Patients see their dermatologist infrequently. This creates large time periods
where skin lesions continue to change, but there is no supervision.

Our system will enable consistent follow-up, potentially allowing cancerous
lesions to be caught earlier. By analyzing lesions over time, we will generate
risk profiles for each patient, bettering clinical decisionmaking. By allowing
the physician to schedule an appointment with a patient after a questionable
lesion is uploaded, our system will promote better patient follow-up.

\section{What Are the Risks and Payoffs?}

\subsection{Risks}
Dermatologists may be hesistant to adopt our technology because in certain
reimbursement schemes, physicians are incentivized only to see the
patient in the office. If our system leads to a decrease in patient visits,
this could slow adoption. Thus, we should focus on hospitals with positive
incentives. Secondly, the ``saturation'' of technology in medicine
\cite{boonstra2014implementing} could mean physicians are unwilling to adopt
additional technology.

\subsection{Payoffs}
The scalability of our system means that it could be deployed to a wide range
of settings, having a large impact on the quality of care. Ultimately, better
patient follow-up could mean that patient outcomes are improved (i.e., lives
saved), practices are managed better, and patients are more satisfied.

\section{How Much Will It Cost? How Long Will It Take?}
Initially, our primary costs are in training neural networks. For our initial
model development, we estimate approximately 150 hours of Amazon instance
utilization ($\sim$\$100 total). Our website will run on a t2.micro instance on
Amazon, which is free under a promotion.

We rely on freely available technologies such as Python, NumPy, TensorFlow
\cite{tensorflow2015-whitepaper}, Caffe \cite{jia2014caffe}, and Ruby on Rails
for the development of our system. From a human resource perspective, we intend
to collectively spend 400 total hours on the project, split amongst model
development, user interface and web application development, user studies,
and report generation, which can be estimated at $\sim$\$40,000 in opportunity
cost with each hour of our time valued at \$100/hour.

\section{How Will Progress Be Measured?}
To measure our progress, we have recruited dermatologists to test our
demonstration system and provide clinical feedback. Moreover, we intend to
perform user studies/focus groups in which patients use our system and offer
feedback. We will administer Likert-scale surveys and analyze results.

We aim to achieve $\geq$90\% in binary skin cancer classification on our test
set. Another checkpoint is the version one application in which patients can
upload images and their physician can view them and interact with the data.
Lastly, in final form there will be patient rewards and an optimized model.

Thus far, Stefano has led model exploration, examining SVM, CNNs, and
pre-trained models \cite{razavian2014cnn}. Thanh has developed the UI and
patient risk scoring. Matt has gathered/cleaned datasets, secured dermatologist
collaborators, and developed the web application. Apurv has done report and
presentation generation.

Future tasks are split as follows. Stefano is focusing on model development and
training (100 hours). Thanh is focused on risk score generation, UI design, and
user studies (100 hours). Matt is building the web application (100 hours).
Apurv is focused on data management and report development (100 hours). All
authors have contributed and will contributed equally in the future to this
work.

\bibliographystyle{abbrv}
\bibliography{proposal}  % sigproc.bib is the name of the Bibliography in this case

\end{document}