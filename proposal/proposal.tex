\documentclass[a4paper,10pt]{article}
\usepackage[utf8]{inputenc}
\usepackage[colorlinks]{hyperref}
\usepackage{epstopdf}
\hypersetup{
  colorlinks = true,
  linkcolor  = magenta,
  citecolor  = magenta
}
\usepackage[all]{hypcap}

%opening
\title{Risk Analysis of Skin Cancer using Deep Learning}
\author{Thanh Dang, Stefano Fenu, Matthew May, Apurv Verma}

\begin{document}

\maketitle

\begin{abstract}
Skin cancer is the most common form of cancer, accounting for 40\% of cases
globally. More than 8,500 people in the US are diagnosed with skin cancer every
day. The survival rate of individuals is dependent on when they start treatment,
and is very high with early treatment. The initial step in the diagnosis of
skin cancer is visual inspection by a trained health care provider. Building on
recent breakthroughs in deep convolutional neural networks, which have proven
to be effective at image classification, we aim to develop a system that
assists dermatology health care providers in following up with their patients.
By providing classification of suspected skin cancer lesions, patient risk
scoring, and patient image management, our system will allow dermatologists
to more effectively manage their patients. This could lead to improved patient
outcomes, enhanced patient satisfication, and streamlined medical practice
management.
\end{abstract}

\section{What Are You Trying To Do?}
Our goal is to build a website that allows dermatologists to more effectively
follow up with patients with skin cancer, or patients at risk for skin cancer.
Our website will do this by providing classification of suspected skin cancer
lesions, patient risk scoring, and patient image analysis over time. Ultimately,
we hope to improve outcomes, promote happier patients, and improve medical
practice.

\section{How Is It Done Today?}
The current state-of-the-art for automated skin cancer analysis generally
involves feature detection \cite{DBLP:journals/eswa/LeeC15}. In dermatology,
there is a widely used ABCDE (Asymmetry, Border, Color, Diameter, Evolving)
\cite{thomas1998semiological} algorithm that is used by doctors to assess
potential skin cancer lesions. Existing approaches often attempt to imitate this
algorithm by ``teaching'' models how to detect the same features
\cite{zagrouba2004prelimary}. However, the limitation of many of these models
has been verification on a relatively small, highly curated test set of images,
leading to low-quality results on broader image sets and limited adoption in
clinical practice.

Relatively recently, neural networks have been applied to skin cancer
classification. In \cite{Kreutz2001}, the authors use a combination of neural
networks, image processing, and feature extraction techniques to classify
skin lesions. Sheha et al. \cite{Sheha_automaticdetection} use a multilayer
perceptron to classify melanoma, attaining 92\% accuracy on their test set.
Esteva et al. \cite{esteva-skincancer-manuscript} use an ensemble of
convolutional neural networks (CNNs) to attain 90\% binary classification
(malignant/benign).

\section{What's New In Your Approach? Why Will It Succeed?}
The primary limitation of current automated approaches has been limited
applicability to actual clinical practice. Low patient engagement, limited
integration into clinical workflows, poor algorithm performance on actual
patient images, and expensive hardware requirements have plagued existing
approaches.

To increase patient engagement, our system will present an interface in which
patients receive rewards for uploading images of their skin lesions
over time, providing an incentive for patient compliance with their treatment
plan.

Due to the recent adoption of electronic medical records (EMR), our Web-based
system will easily integrate with existing clinical workflows, which are
increasingly electronic. Our system will perform analysis on uploaded patient
images and then present the results to the physician in a meaningful,
uncluttered way, allowing immediate follow-up with the patient if necessary.

To improve algorithm performance, we will use state-of-the-art convolutional
neural networks, which have been shown to be highly effective
\cite{krizhevsky2012imagenet} at difficult image classification problems.
Furthermore, by allowing patients to upload images of their skin lesions from
their smartphone or computer, there will be no additional hardware purchase
required of the patient or physician.

\section{Who Cares?}
In the literature, it has been shown that the implementation of health care
IT can lead to significant increases in patient satisfaction
\cite{roham2012predicting}. It has also been shown that the experience of the
patient in receiving care is related to overall patient
satisfaction \cite{bjertnaes2012overall}. Patient satisfaction is one of the
primary metrics now considered an indicator of overall health care quality by
physicians and hospital administrators \cite{fenton2012cost}.

By allowing patients to receive a higher standard of care through the use of
unobtrusive technology, we believe patient satisfaction will grow.
Furthermore, it has been shown that telemedicine technology, which this could be
considered a form of, is effective and increases patient access to care
\cite{hilty2013effectiveness}. It also allows for more efficient use of
resources, as physicians can interact with their patients in an asynchronous
fashion.

In our research, we have often heard that physicians are frustrated by
non-compliant patients. By creating a communication tool that connects the
physician with his patients, and provides meaningful data that supports
patient follow-up, we believe patient outcomes, as well as patient satisfaction
will be improved.

\section{What Difference Will It Make?}
To determine whether our proposed approach would have utility in practice, we
spoke with dermatologists at Stanford Medical Center and Emory University. One
common thread that we heard was that patient follow-up is a process that has
significant gaps. Patients see their dermatologist at six-month, or in some
cases, multi-year intervals, and have little to no contact with their physician
in between. This creates large time periods where skin lesions continue to
change, but there is little or no supervision of their progress.

Our system will enable follow-up on a consistent basis when the patient is not
physically in the office, potentially allowing cancerous lesions to be caught
at a much earlier time. Furthermore, by analyzing skin lesions over time, we
will be able to generate risk profiles for each patient, increasing the
quality of clinical decisionmaking. And by allowing the physician to instantly
``ping'' the patient to schedule an appointment after a questionable lesion
is uploaded, our system will promote more frequent patient follow-up when
necessary.

\section{What Are the Risks and Payoffs?}

\subsection{Risks}
As with any technology, there are certain risks that should be considered.
First, dermatologists may be hesistant to adopt our technology because in
certain reimbursement schemes, physicians are compensated primarily when
patients come for in-person office visits. If our system leads to a decrease
in patient visits because of more effective patient management, this could
slow adoption. To mitigate this, we should focus on health care systems with
positive incentives for managing patients more effectively.

Secondly, the recent widespread adoption of electronic medical records is
a relatively costly and complex procedure that can be frustrating for
physicians \cite{boonstra2014implementing}. Thus, some medical practices may
be saturated with technology and unwilling to adopt additional technologies.
Lastly, on a more technical note, the relatively small size of our image
datasets used for training and testing could promote overfitting, leading
to suboptimal results when applied to patient imaging in practice.

\subsection{Payoffs}
The scalability of our system, and the relatively low cost that it could be
offered to medical practices at, means that it could feasibly be deployed to a
wide range of clinical settings, having a large impact on the quality of medical
care. Ultimately, better patient follow-up could mean that patient outcomes are
improved (i.e., lives saved), practices are managed more effectively, and
patients are more satisfied with the care they've received, making them more
likely to return.

\section{How Much Will It Cost? How Long Will It Take?}
Our approach leverages open-source technology and also a novel technique known
as \textit{transfer learning} \cite{razavian2014cnn}, in which a pre-trained
network is used at the initial basis for feature learning. Thus, we are able to
produce a minimally viable product at relatively low cost. More specifically,
our primary costs are in training neural networks, which are done at a cost of
$\sim$\$0.65/hour on Amazon EC2 dedicated GPU instances. For our initial model
development and training, we estimate approximately 150 hours of instance
utilization ($\sim$\$100 total). Our website will run on a t2.micro instance on
Amazon EC2, which is free under a promotion. If the promotion were to expire,
we estimate we could operate the server initially at $\sim$\$0.013/hour, or
about \$10/month.

We rely on freely available technologies such as Python, NumPy, TensorFlow
\cite{tensorflow2015-whitepaper}, Caffe \cite{jia2014caffe}, and Ruby on Rails
for the development of our system. From a human resource perspective, we intend
to collectively spend 400 total hours on the project, split amongst model
development, neural network training, user interface development, and user
studies, which can be estimated at $\sim$\$40,000 in opportunity cost with
each of our time valued at an average of \$100/hour.

\section{How Will Progress Be Measured?}
To measure our progress, we have recruited dermatologists from two leading
institutions, to test our demonstration system and provide feedback from a
clinical perspective. Moreover, we intend to perform simple user studies/focus
groups in which we recruit patients to use our system and offer feedback on
their experience and the likelihood they would use it in reality.

From a more technical perspective, we aim to achieve at least 90\% in binary
skin cancer classification on our test image set. Another checkpoint is the
launch of our application as a minimal product in which patients can upload
images and their physician can view them as a stream and interact with the
data. Lastly, in final form we intend to provide additional user interface
enhancements such as patient rewards for uploads and development of the model
using an ensemble approach.

Thus far, Stefano has led model exploration and training, examining a wide range
of approaches including simple SVM classifiers (baseline), deep CNNs, and
pre-trained models. Thanh has developed the user interface and researched
patient risk scoring. Matt has prepared our training and test datasets,
secured dermatologist collaborators, and developed the initial web application
backend. Apurv has done report and presentation generation, and surveyed the
relevant literature for prior art.

Going forward, the tasks are split as follows among the team members.
Stefano is focusing on model selection, development, and training (100 hours).
Thanh is focused on risk score generation for patients, UI design, and user
studies (100 hours). Matt is building the web application and handling
integration with the model API (100 hours), as well as interfacting with our
dermatologist collaborators. Lastly, Apurv is focused on data pre-processing,
component integration, and report and presentation development (100 hours). All
authors have contributed and will contributed equally in the future to this
work.

\bibliographystyle{abbrv}
\bibliography{proposal}  % sigproc.bib is the name of the Bibliography in this case

\end{document}