\documentclass[a4paper,10pt]{article}
\usepackage[utf8]{inputenc}

%opening
\title{Risk Analysis of Skin Cancer using Deep Learning}
\author{Matthew May, Stefano Fenu, Thanh Dang, Apurv Verma}

\begin{document}

\maketitle

\begin{abstract}
Skin cancer is the most common form of cancer, accounting for 40\% of cases globally.
More than 8,500 people in the US are diagnosed with skin cancer everyday. Melanoma rates in the United States have doubled from 1982 to 2011.
The survival rate of people with it depends on when they start treatment. The cure rate is very high
with early prognosis. The curability is as high as 92\% if the cancer is detected early. The prognosis is less favorable 
if it has spread to other parts of the body.
Deep convolutional neural networks have led to new breakthroughs in image classification.
Advancement in neural network frameworks and GPU based computing has allowed us to train deeper and deeper networks.
We aim to exploit these advancements and apply it to the problem of early skin cancer detection.
\end{abstract}

\section{Problem Statement}
To build a platform where dermatologists can easily scan through the cancer prone patients. The system uses deep learning
techniques in the background to learn from thousands of images and associates a risk score with each uploaded image. The value of the
score can be used to differentiate between malign and benign cases. We also aim to train a model for multi-class classification among the
23 classes.

\section{Current Approaches/Survey}
It is an unfortunate fact of the healthcare system that it spends more in research on expensive treatments for late-stage diseased individuals
than developing scalable and cost-effective early screening methods. To the best of our knowledge, no one has attempted to build such a system.

It is only recently that neural networks have been used for the problem of skin cancer classification. Much of the earlier work focussed
on extracting features based on image processing techniques to do the classification. In \cite{DBLP:journals/eswa/LeeC15}, Lee and Chen use low
level feature comparison for detecting different cancer types. In \cite{Kreutz2001}, the authors use a combination of neural networks and 
sophisticated image processing and feature extraction techniques to achieve a fast and reliable diagnosis. They were able to obtain a 
90\% correct classification of malignant and benign skin lesions from the DANAOS data collection. In \cite{Sheha_automaticdetection} the authors use 
a multilayer perceptron to classify melanoma. 

There have also been techniques like \cite{Gniadecka2004443}, \cite{Zhao:15} which combine spectroscopy data with neural networks. These techniques
are only plausible in a lab setting with appropriate instruments. The novelty of our technique is that it relies solely on image data.
These images can be captured by an individual using smartphones and does not require any specialized pathology skills. The user could get a reminder
to upload the image to the website on a periodic basis and the dermatologist could monitor high risk patients easily using our system.\\
The most recent work which is related to our current work was done by Andre, Kuprel and Sebastian Thrun at Stanford. \cite{esteva-skincancer-manuscript}.
The approach used by them is an ensemble of 5 models (VGG16-1, VGG16-2, VGG16-3, VGG19-1, and VGG19-2) that classify among the 23 classes. We try to
achieve a better accuracy over their model and also build a system for dermatologists to efficiently manage their patients.

\section{Salient Features and Innovation}
\begin{itemize}
 \item A model which can work with images captured through webcams, smartphones or other day-to-day shops medical devices.
 \item An interface for patients to upload their image periodically so the clinician can monitor the lesion outside the clinic
 \item An interface for the clinician to view high risk patients at the top of their feed. The high risk patients are found out by passing the
       uploaded image through the learnt network.
 \item The system also displays similar lesions as to provide an explanation of why it considers the lesion to be of that particular kind.
 \item Deep convolutional networks are the state of the art in image classification and get better and better with more training data. So
 far we have tried AlexNet\cite{NIPS2012_4824}, GoogleNet\cite{43022} and are currently experimenting with ResNet\cite{DBLP:journals/corr/HeZRS15}.
 If time permits we want to try out Net2Net \cite{DBLP:journals/corr/ChenGS15} which provides an efficient way to transfer knowledge from one neural
 network to another.
\end{itemize}

\section{Users}
The tool can be used by dermatologists and patients.

\section{Impact}
The tool provides a low cost, scalable approach to detect skin cancer early by empowering patients and dermatologists and providing
efficient and timely detection. We are in talks with a couple of dermatologists and would incorporate their feedback in our design.
Further, we will be using cross validation to measure the accurancy of our neural networks.\\
If we are successful, we could reduce the thousands of death tolls to skin cancer each year. We also plan to host the
website public and do beta testing by giving access to selective patients and health professionals.

\section{Risks and Payoffs}
\begin{itemize}
 \item Uneasiness of adoption by dermatologists because of re-imbursement structure in healthcare. Dermatologists get paid every time a patient
 walks in to their office.
 \item Dermatologists also do not want to use more technology as they are already saturated with technologies.
 \item Cold start problem: Lack of training data for a few classes initially, might cause the network to have bad classification accuracy for a
 few classes. This might lead to dermatologists losing trust in the system.
\end{itemize}

\section{Costs and Timeline}
The overall approach is very cost-effective. We rely completey on open source technologies like tensorflow\cite{tensorflow2015-whitepaper} and caffee\cite{jia2014caffe} for our building and training
deep neural network. We use Ruby on Rails for building the web application which is also open source.Typical training and fine-tuning of parameters
for the model would require ~100 hours of training time.  Amazon dedicated GPU instance
typically cost ~\$0.65/hour
\begin{itemize}
 \item Explore and experiment various convolutional neural network models (1-2 months)
 \item Build the web application (1-2 months). We plan to do these two steps in parallel
 \item Final report and evaluation and gathering feedback. (2 weeks)
\end{itemize}


\section{Evaluation and Checkpoints}

\begin{enumerate}
 \item We aim to achieve an accuracy of 90\% for the binary classification case.
 \item Build an end to end web application with user and clinician interfaces.
 \item Build a service which can compute score for each uploaded image
\end{enumerate}
The tasks are split as follows among the team members. Stefano is mainly focussing on training the deep network. Thanh will be focussing on 
building the UI and visualization component. Matthew will be focussing on the web application and Apurv would be focussing on data pre-processing
and integrating the components.

\bibliographystyle{abbrv}
\bibliography{proposal}  % sigproc.bib is the name of the Bibliography in this case

\end{document}
